\documentclass[a4paper, 12pt, brazilian]{article}
\usepackage[T1]{fontenc}

\usepackage{amsmath, amsfonts, amssymb}
\usepackage{siunitx}

\usepackage[
top=2cm,
right=3cm,
bottom=2cm,
left=3cm
]{geometry}

\usepackage{xcolor}
\usepackage{tikz}
\usepackage{import}
\usepackage{float}

\usepackage{graphicx}
\usepackage{hyperref}
\usepackage{cleveref}
\usepackage{bm}
\usepackage{multirow}
\usepackage{cancel}
\usepackage[version=4]{mhchem}

\usepackage{config}

\begin{document}
	\section{Primeira questão}
	Calcule a energia fornecida pela bomba ($h_{b}$ em \SI{}{\meter}, com duas casas decimais) considerando o esquema abaixo. O aspersor localizado no ponto 2 opera com pressão de $\SI{4}{\kilogram f/\centi\meter^{2}}$ e a vazão que escoa na canalização é de $\SI{10}{\meter^{3}/\hour}$. O diâmetro da tubulação é de \SI{50}{\milli\meter} e a perda de carga total entre os pontos 1 e 2 é de \SI{15}{\meter}. A carga cinética no ponto 1 é desprezível e este ponto é uma superfície livre sujeita a pressão atmosférica. A cota em 1 vale \SI{100}{\meter} e em 2 vale \SI{146.3}{\meter}.
	\begin{center}
		\includegraphics[width=.9\linewidth]{assets/images/ex1}
	\end{center}
	\subsection{Solução}
	A energia fornecida pela bomba pode ser dada pela equação de Bernoulli modificada como segue
	\begin{equation}
		\label{eq:bernoulli}
		Z_{1}+\dfrac{v_{1}^{2}}{2g}+\dfrac{p_{1}}{\gamma}+h_{b}=Z_{2}+\dfrac{v_{2}^{2}}{2g}+\dfrac{p_{2}}{\gamma}+hf_{1-2}
	\end{equation}
	A partir do que é dito no enunciado algumas simplificações podem ser feitas na equação anterior. Ao mudar o referencial das cotas é possível desprezar $Z_{1}$ fixando $Z_{2}=\SI{46.3}{\meter}$. No ponto 1 é cabível desprezar a pressão atuante, já que somente as moléculas da atmosfera agem. A carga cinética também é desprezada. Do lado direito da Equação de Bernoulli todos os termos serão considerados, só que para haver a substituição dos valores visando calcular $h_{b}$ as devidas conversões devem ser feitas para o Sistema Internacional.
	\begin{center}
		\includegraphics[width=.9\linewidth]{assets/images/referencia_1}
	\end{center}
	Para a pressão, é sabido que \SI{1}{\kilogram f} corresponde a \SI{9.81}{\newton}, então
	\begin{eqnarray}
		p_{2}&=&\SI{4}{\dfrac{\kilogram f}{\centi\meter^{2}}}\\
			 &=&\dfrac{4\cdot 9.81}{(10^{-2})^{2}}\SI{}{\dfrac{\newton}{\meter^{2}}}\\
			 &=&\SI{392400}{\pascal}
	\end{eqnarray}
	A vazão dada em $\SI{}{\meter^{3}/\hour}$ deve ser convertida para $\SI{}{\meter^{3}/\second}$ como segue
	\begin{eqnarray}
		Q_{2}&=&\SI{10}{\dfrac{\meter^{3}}{\hour}}\\
			 &=&\dfrac{10}{3600}\SI{}{\dfrac{\meter^{3}}{\second}}\\
			 &=&\SI{2.777e-3}{\meter^{3}/\second}
	\end{eqnarray}
	Após obter a vazão $Q_{2}$, para calcular a velocidade é preciso considerar a fluxo de água que atravessa a seção transversal do tubo como é descrito pela equação
	\begin{equation}
		Q_{2}=v_{2}\cdot A
	\end{equation}
	assim
	\begin{eqnarray}
		Q_{2}&=&v_{2}\cdot\dfrac{\pi d^{2}}{4}\Rightarrow\\
		\Rightarrow	v_{2}&=&\dfrac{4Q_{2}}{\pi d^{2}}
	\end{eqnarray}
	como $d_{2}=\SI{50}{\milli\meter}=\SI{.05}{\meter}$, vem
	\begin{eqnarray}
		v_{2}&=&\dfrac{4\cdot 0.00277}{\pi\cdot 0.05^{2}}\\
			 &=&\SI{1.411}{\meter/\second}
	\end{eqnarray}
	Substituindo em \eqref{eq:bernoulli}
	\begin{eqnarray}
		h_{b}&=&Z_{2}+\dfrac{v_{2}^{2}}{2g}+\dfrac{p_{2}}{\gamma}+hf_{1-2}\\
			 &=&46.3+\dfrac{1.411^{2}}{2\cdot 9.81}+\dfrac{392\,400}{9\,810}+15\\
			 &=&\SI{101.40}{\meter}
	\end{eqnarray}
	\section{Segunda questão}
	Após percorrer o trecho vertical $AB$, a água descarrega em forma de jato na atmosfera, como mostra a figura abaixo. Sabendo que o diâmetro do tubo $A$ é quatro vezes maior que o de $B$ e que a pressão no ponto $A$ é de $\SI{.58}{\kilogram f/\centi\meter^{2}}$, estime a altura $H$ do jato (resultado em \SI{}{\meter}, com duas casas decimais), desprezando as perdas de carga e as perdas devido ao atrito com o ar. O fluido é água ($\gamma_{\ce{H2O}}=\SI{1000}{\kilogram f/\meter^{3}}$)  e a distância vertical entre $A$ e $B$ vale \SI{.3}{\meter}.
	\begin{center}
		\includegraphics[width=.3\linewidth]{assets/images/ex2}
	\end{center}
	\subsection{Solução}
	O primeiro trecho analisado estabelece a conservação de energia entre as cotas $A$ e $B$. Ao escrever a equação de Bernoulli nesse caso, obtemos
	\begin{equation}
		Z_{A}+\dfrac{v_{A}^{2}}{2g}+\dfrac{p_{A}}{\gamma_{\ce{H2O}}}=Z_{B}+\dfrac{v_{B}^{2}}{2g}+\dfrac{p_{B}}{\gamma_{\ce{H2O}}}+hf_{A-B}
	\end{equation}
	A partir da figura e das considerações feitas no enunciado podemos cancelar os seguintes termos
	\begin{equation}
	\cancel{Z_{A}}+\dfrac{v_{A}^{2}}{2g}+\dfrac{p_{A}}{\gamma_{\ce{H2O}}}=Z_{B}+\dfrac{v_{B}^{2}}{2g}+\cancelto{p_{B}=p_{\text{atm}}}{\dfrac{p_{B}}{\gamma_{\ce{H2O}}}}+\cancel{hf_{A-B}}
	\end{equation}
	logo
	\begin{equation}
	\label{eq:bernoulli2}
	\dfrac{v_{A}^{2}}{2g}+\dfrac{p_{A}}{\gamma_{\ce{H2O}}}=Z_{B}+\dfrac{v_{B}^{2}}{2g}
	\end{equation}
	Como o fluxo que atravessa a seção transversal da tubulação em $A$ e $B$ é o mesmo, é possível determinar a relação entre as velocidades e denotar $v_{B}$ como função de $v_{A}$, assim
	\begin{eqnarray}
		Q_{A}&=&Q_{B}\\
		v_{A}\,A_{A}&=&v_{B}\,A_{B}\\
		v_{A}\,\dfrac{\bcancel{\pi}d_{A}^{2}}{\cancel{4}}&=&v_{B}\,\dfrac{\bcancel{\pi} d_{A}^{2}}{\cancel{4}}\\
		v_{A}\,d_{A}^{2}&=&v_{B}\,d_{B}^{2}
	\end{eqnarray}
	Sendo $d_{A}=4\,d_{B}$, vem
	\begin{eqnarray}
		v_{A}\,(4d_{B})^{2}&=&v_{B}\,d_{B}^{2}\\
		v_{A}&=&\dfrac{v_{B}}{16}
	\end{eqnarray}
	Retornando em \eqref{eq:bernoulli2} e substituindo $v_{A}$
	\begin{eqnarray}
		\dfrac{(v_{B}/16)^{2}}{2g}+\dfrac{p_{A}}{\gamma_{\ce{H2O}}}&=&x+\dfrac{v_{B}^{2}}{2g}\\
		\dfrac{v_{B}^{2}}{512g}+\dfrac{p_{A}}{\gamma_{\ce{H2O}}}&=&x+\dfrac{256\,v_{B}^{2}}{512g}\\
		\dfrac{p_{A}}{\gamma_{\ce{H2O}}}&=&x+\dfrac{255\,v_{B}^{2}}{512g}\\
		\therefore v_{B}^{2}&=&\left(\dfrac{p_{A}}{\gamma_{\ce{H2O}}}-x\right)\dfrac{512g}{255}\label{eq:squaredvb}
	\end{eqnarray}
	Após obter uma relação para $v_{B}$, ao tomar como base as cotas em $B$ e $H$ e aplicar Bernoulli novamente
	\begin{equation}
			Z_{B}+\dfrac{v_{B}^{2}}{2g}+\dfrac{p_{B}}{\gamma_{\ce{H2O}}}=Z_{H}+\dfrac{v_{H}^{2}}{2g}+\dfrac{p_{H}}{\gamma_{\ce{H2O}}}+hf_{B-H}
	\end{equation}
	cancelando os termos pertinentes, temos
	\begin{equation}
		\cancel{Z_{B}}+\dfrac{v_{B}^{2}}{2g}+\cancel{\dfrac{p_{B}}{\gamma_{\ce{H2O}}}}=Z_{H}+\cancel{\dfrac{v_{H}^{2}}{2g}}+\cancel{\dfrac{p_{H}}{\gamma_{\ce{H2O}}}}+\cancel{hf_{B-H}}
	\end{equation}
	então
	\begin{eqnarray}
		\label{eq:bernoulliBtoH}
		\dfrac{v_{B}^{2}}{2g}&=&Z_{H}
	\end{eqnarray}
	Substituindo \eqref{eq:squaredvb} em \eqref{eq:bernoulliBtoH}
	\begin{eqnarray}
		\left(\dfrac{p_{A}}{\gamma_{\ce{H2O}}}-x\right)\dfrac{512\cancel{g}}{255}&=&2\,\cancel{g}\,Z_{H}\\
		Z_{H}&=&\left(\dfrac{p_{A}}{\gamma_{\ce{H2O}}}-x\right)\dfrac{512}{510}\label{eq:zh}
	\end{eqnarray}
	A pressão em $A$ deve ser convertida para Pascal (Pa)
	\begin{eqnarray}
		p_{A}&=&\SI{.58}{\kilogram f/\centi\meter^{2}}\\
			 &=&\dfrac{0.58\cdot 9.81}{(10^{-2})^{2}}\SI{}{\dfrac{\newton}{\meter^{2}}}\\
			 &=&\SI{56989}{\pascal}
	\end{eqnarray}
	e o peso específico em $\SI{}{\newton/\meter^{3}}$
	\begin{eqnarray}
		\gamma_{\ce{H2O}}&=&\SI{1000}{\kilogram f/\meter^{3}}\\
						 &=&1000\cdot 9.81\,\SI{}{\newton/\meter^{3}}\\
						 &=&\SI{9810}{\newton/\meter^{3}}
	\end{eqnarray}
	como $x=\SI{.3}{\meter}$, após substituir em \eqref{eq:zh} é obtido $Z_{H}=H$
	\begin{eqnarray}
		Z_{H}&=&\left(\dfrac{56\,989}{9\,810}-0.3\right)\dfrac{512}{510}\\
			 &=&\SI{5.53}{\meter}
	\end{eqnarray}
\end{document}