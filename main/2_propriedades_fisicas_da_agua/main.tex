\documentclass[
	a4paper, 
	12pt, 
	brazilian
]{article}

\usepackage[utf8]{inputenc}
\usepackage[T1]{fontenc}

\usepackage{amsmath, amsfonts, amssymb}
\usepackage{siunitx}

\usepackage[
top=2cm,
right=3cm,
bottom=2cm,
left=3cm
]{geometry}

\usepackage{xcolor}
\usepackage{tikz}
\usepackage{import}
\usepackage{float}

\usepackage{graphicx}
\usepackage{hyperref}
\usepackage{cleveref}
\usepackage{bm}

%\newcommand{\txt}[1]{\textrm{#1}}

\usepackage{config}

\begin{document}
	\section{Propriedades físicas da água}
	
	\begin{itemize}
		\item\textbf{Massa específica}
			\begin{equation}
				[\,\rho\,]=\left[\,\dfrac{m}{v}\,\right]=\SI{}{\dfrac{\kilogram}{\meter^{3}}}
			\end{equation}
			$\SI{4}{\SIUnitSymbolCelsius}\rightarrow\rho=\SI{1000}{\dfrac{\kilogram}{\meter^{3}}}$
			
			A \textbf{densidade} da água é uma função de sua \textbf{temperatura}. Para outros fluidos as tabelas devem ser consultadas.	
			
			\textbf{Equação de Tanaka (2001):}
			Equação utilizada para determinar a densidade da água para variações de temperatura entre \SI{0}{\SIUnitSymbolCelsius} e \SI{40}{\SIUnitSymbolCelsius}.
			
			\begin{equation}
			\rho=999.974950\cdot\left(1-\dfrac{(T-3.983035)^{2}\cdot(T+301.797)}{522528.9\cdot (T+69.34881)}\right)
			\end{equation}	
		\item\textbf{Peso específico}
			\begin{equation}
				\gamma=\dfrac{W}{v}=\dfrac{m\cdot g}{v},\;\textrm{sendo}\;g=\SI{9.81}{\meter/\second^{2}}
			\end{equation}
			\begin{equation}
				[\,\gamma\,]=\SI{}{\dfrac{\newton}{\meter^{3}}}
			\end{equation}	
			A água a \SI{4}{\SIUnitSymbolCelsius} possui $\gamma=\SI{9810}{\newton/\meter^{3}}=\SI{1000}{kgf/\meter^{3}}$
			$\gamma$ pode ser reescrito como
			\begin{equation}
				\gamma=\rho\cdot g
			\end{equation}
		\item\textbf{Densidade (adimensional)}
		\begin{equation}
			d=\dfrac{\rho_{\textrm{substância}}}{\rho_{\textrm{padrão}}}
		\end{equation}
		A água a \SI{4}{\SIUnitSymbolCelsius} tem $d=1$, sendo utilizado como o padrão. Nessa temperatura $\rho_{\textrm{padrão}}=\SI{1000}{\kilogram/\meter^{3}}$.
		
		\item\textbf{Viscosidade:}
		\begin{itemize}
			\item[\textbf{(a)}] Coeficiente de viscosidade dinâmica
			
			Propriedade que confere resistência ao cisalhamento
			\begin{equation}
			\mu=f(\textrm{fluido, temperatura})
			\end{equation}
			\begin{equation}
			[\,\mu\,]=\SI{}{\pascal\cdot\second}
			\end{equation}
			Água a \SI{4}{\SIUnitSymbolCelsius} possui $\mu=\SI{1.566e-3}{\pascal\cdot\second}$
			
			\textbf{Equação de Lichachev (2003):}
			Calcula $\mu$ em função de $T$
			\begin{equation}
			\mu=32.025666\cdot 10^{-6}\cdot e^{\frac{482.134866}{T+119.886026}}
			\end{equation}
			
			\item[\textbf{(b)}] Coeficiente de viscosidade cinemática
				\begin{equation}
					\nu=f(\textrm{fluido, temperatura})
				\end{equation}
				\begin{equation}
					[\,\nu\,]=\left[\dfrac{\mu}{\rho}\right]=\SI{}{\dfrac{\meter^{2}}{\second}}
				\end{equation}
				\import{assets/tables}{viscosidade_cinematica_temperatura}
		\end{itemize}
	\end{itemize}
\end{document}
